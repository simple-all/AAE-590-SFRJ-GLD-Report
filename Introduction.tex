\section{Introduction}

\subsection{Mission Objective - Melanie Grande}
\lettrine{S}{olid} fuel ramjets present an opportunity to realize strategic performance enhancements through use of innovative bypass strategies, fuel loading, and other design characteristics. This study from a design team at Purdue University presents a detailed conceptual design, against which trades can be made to further improve performance and increase ground range. The design methods draw on the knowledge published in academic and military literature and add unique features for a solid fuel ramjet (SFRJ) Ground Launch Demonstrator (GLD). The purpose of the study was to evaluate the feasibility of an SFRJ with the objective to maximize ground range. 

This paper will review the design methods and simulation of each SFRJ subsystem, including the chin-style inlet design, the AeroValve pattern, the combustor and fuel grain, fin mechanisms, and more. A simulator environment was created in order to integrate the subsystems, model the trajectory, and evaluate performance as the design evolved. The simulator was able to demonstrate the feasibility of the SFRJ as well as perform key design trades and parameter sweeps. Trajectory and performance analysis was necessary to fully understand the impact of various decision decisions on the mission.

The GLD concept was subject to a number of system-level requirements presented to the design team. The GLD should include an off-the-shelf, solid motor boost stage, and during its burn, the SFRJ stage should make two turns up to 45\textdegree. The baseline fuel should be hydroxyl-terminated polybutadiene (HTPB), but the design team should explore additives such as Boron-loading for performance improvements. The SFRJ should use a chin-type inlet and assess air bypass strategies with associated valving and controlling for effects on performance and operability. Also, the airframe outer diameter should be set to match the diameter of the solid booster, and only the fins may extend beyond this dimension. Finally, the vehicle must withstand all flight loads with a factor of safety of 1.5, and the system should be designed within the manufacturing capabilities at Purdue University.

\subsection{Literature Review}
Brief overview of what's been done in the past for SFRJs and state of the art. Compare to what we're trying to accomplish.
\color{red}Would it be better to just discuss what we learned from literature in the various subsections?\color{black}