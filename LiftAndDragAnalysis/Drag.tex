\subsubsection{Drag Analysis - Manav Vaghashiya}
Drag analysis is an important aspect of verifying the feasibility of the SFRJ design. Validation of the drag coefficient can be done by comparing the analysis at a specified Mach number against an existing design. This can provide insight into whether the design will satisfy its requirements and specifications. It is important to have an accurate and realistic drag coefficient since it affects the range of the SRFJ as well as other performance parameters. The complex design of the SFRJ posed an intricate challenge when conducting the analysis.

Drag analysis of the vehicle is divided into several components. Drag coefficients from Skin Friction Drag, Wave Drag, Body-Base Drag, Fin Drag, Inlet Cowl Drag and Inlet Spillage Drag are amalgamated to form the complete drag analysis of the vehicle. Each component or components is calculated using various sources of literature. Skin Friction Drag, Wave Drag and Body-Base Drag is calculated using literature from \textit{Tactical Missile Design} \cite{fleeman_2001}. Inlet Cowl Drag is calculated using “A Tool for the Aerodynamic Design and Analysis of Supersonic Inlets” and Inlet Spillage Drag is computed using “Supersonic Inlet Design.”

Skin Friction Drag is the drag created by the friction of air against the surface of the SFRJ. Wave Drag is the drag imposed on the SFRJ by the formation of shocks. Body- Base drag is the drag created tail end of the missile. Skin Friction Drag, Wave Drag and Body-Base drag is computed using several loops which consider the geometry of the vehicle as well as flow properties such as velocity. Inlet Cowl Drag is the drag caused by the shock created at the cowl and Inlet Spillage Drag is created by the “spillage” of air at the inlet of the SFRJ. Both, Inlet Cowl Drag and Inlet Spillage Drag is computed using 2-dimensional equations from literature that consider the geometry of the inlet and the flow properties at the inlet. Fin Drag is the drag produced by the fin and is calculated using CFD analysis of the fin and interpolating the data. 
