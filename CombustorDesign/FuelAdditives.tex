\subsection{Fuel Additives} 
\subsubsection{Background and Motivation}
The baseline fuel for design of the vehicle was neat HTPB, however Boron as a fuel additive was investigated for further performance improvement. Boron was chosen for further investigation due to its extremely high volumetric energy density and low toxicity compared to many conventional modern propellants. \\ \indent
Boron and its derivatives have been investigated for use in rockets since the 1950's, however neat boron has been plagued by ignition and combustion difficulties. John D. Clark, author of \textit{Ignition!: An informal history of rocket propellants}, was once asked for his definition of an exotic fuel to which he responded "It's expensive, it's got boron in it, and it probably doesn't work". However, recent advancements and new thinking, principally in the addition of the AeroValve, has improved our ability to burn Boron and advanced the state of Born combustion research. \\ \indent
The earliest recorded use of Boron in a ducted rocket flight vehicle was from the Augmented Thrust Propulsion (ATP) missile program from 1965-1971 [Keirsey]. Fuel loading's up to 60\% were investigated with claimed efficiencies as high as 90\%. The system was constructed from four individual inlets, combustors and exit nozzles. A single propellant grain was burned in a center blast tube and fed each combustor with hot boron-rich flows. 
\\ \indent
The United States Air force continued Boron solid fuel testing at the component level with the Boron Solid Fuel Ramjet (BSFRJ) and Boron Engine Component Integration, Test and Evaluation (BECITE) programs. Germany has also investigated Boron for flight vehicles, using a Boron-based fuel on the Anti Radiation Missile with Intelligent Guidance and Extended Range (ARMIGER). All of these vehicles, with the exception of BSFRJ, have been based on the Integral Rocket Ramjet model (IRR). An IRR combines the boost propellant chamber and ramjet fuel in a single continuous combustor which aids the combustion of Boron. The IRR chamber extends the mixing region, and in some vehicles provides extra burning oxidizer from the boost motor grain which significantly helps with combustion of the boron particles. \\ \indent
Literature indicates that the principle challenge in sustained ignition of boron particles occurs due to the two separate stages of combustion. First the particle ignites and the layer of $B_2O_3$ is removed, then the exposed born layer is burned. For optimal performance efficiency, the $B_2O_3$ must be recodensed before the nozzle. The performance limits lend themselves to needing three distinct conditions.
\begin{enumerate}
    \item High temperature initial to shed to outer layer of $B_2O_3$.
    \item High O/F to burn the exposed boron
    \item Low enough temperature to recondense the $B_2O_3$ back into a liquid state
\end{enumerate}
Conditions 1 conflict with conditions 2 and 3, as O/F increases from optimal temperature will decrease. \\ \indent
There exists some debate within the SFRJ community as to which global reaction rate model to use and two primary models have been developed and discussed. The first model, proposed by King in 1972, and the second model, originally proposed by Glassman and updated by Li and Williams in 1990, primarily differ in how they consider the heterogeneous reactions occurring on the particle surface. We chose to work with the King model, to align results with earlier analytical and experimental work by Benni Natan at Technion.
\subsubsection{Combustion Model}
In order to evaluate Boron as a potential SFRJ additive, a method of determining the combustion efficiency is required. We developed a code which calculates the reaction rate of a Boron particle in a SFRJ flow field. The 1D model considering the effect of flow temperature and composition to determine the diameter of a Boron particle at a specific axial location and moment in time. \\ \indent
Natan notes that there is a strong dependence on the ejection velocity of the fuel surface due to the particle crossing the flame boundary zone, and thus built his model in 2D to account for this. Due to computational limits, we limited our analysis to 1D. We assumed that the initial temperature of the Boron particle was 800$^{\circ}$ which account for preheating of the fuel before the Boron particle is entrained in the flow field. We also assumed that a released particle would start at 5 m/s and aerodynamic effects would increase particle speed until it matched the gas flow speed. However, the model is very insensitive to this initial speed because the released particle reaches flow velocity very quickly. Lastly, the flow temperature of Boron was calculated in CEA assuming complete combustion. Although CEA is not an especially useful tool with Boron because it cannot account for unburned products, it did allow us to calculate an approximate gas temperature. \\ \indent
The governing equations were based on the work of King and primarily considered the thermolysis and hydrolysis of the $B_2O_3$ shell, and combustion of the exposed Boron with $O_2$ in the chamber. 
\subsubsection{Bypass Usage}
Due to the aforementioned complications with Boron combustion, innovative solutions are required. One such solution involves a bypass flow redirected and dumped directly in an aft mixing chamber. This bypass flow has two purposes, firstly it ensures that any unburned Boron particles have ample $O_2$ to complete combustion. And secondly, it cools the particles enough to recondense $B_2O_3$ and ensure that maximum energy is extracted from the fuel formulation. However, this method requires a larger inlet in order to justify the high percentage of flow not being combusted. \\ \indent
In order to improve the bypass air concept, the team created the AeroValve which redirects bypass air into the chamber at discrete points. This allows further control of the $O_2$ content withing the chamber. With proper optimization of valve control, a high percentage of the boron could be burned and $B_2O_3$ recondensed in order to improve performance. \\ \indent
\subsubsection{Manufacturing}
\subsubsection{Results}
Two sets of results were produced which demonstrate the efficiency of combustion in the chamber and a second set which demonstrate the theoretical performance of Boron in the flight vehicle. Future work to integrate the models and determine the actual performance is required. \\ \indent
As seen below in figure
\subsubsection{Future Work}
Experimental work is required to further the development of additives in SFRJ's. Though considerable issues occur with boron combustion, experiments with new bypass techniques such as the AeroValve, may be able solve these problems. \\ \indent
Further investigation into other additives to improve performance. The SFRJ did not suffer from a low thrust force as is often seen in hybrid rockets, so additives which improve regression will not help performance as much as those that improve Isp. \\ \indent
An additional option is investigate oxidizer additives to improve the oxidizer content in the chamber. If added in low enough quantities to not be self-sustaining, they present a safe and economic method to improve performance.